Michael Gladych: A View of a Science Journalist as Science History
Initial ariticle synopsis
Hit upon his history of writing about science both from the fringe and the mainstream.  Show how they interweave and how they paint a picture of science in the ‘50s and ‘60s.  Describe Gladych’s life and how he became part of the fringe through Project Artichoke.  NOTE:  Leave Project Artichoke out of this piece.
<needs a paragraph here introducing the premise for the article>
A look at scientific research in post-war America during the 1950s and ‘60s as seen through the eyes and life of science journalist Michael Gladych reveals a time when scientific possibilities were grander and fairly dripped with the promise of sci-fi style adventure.
Michael Gladych first caught my eye in a quote from “The Hunt for Zero Point”, a fringe physics classic, by Nick Cook: 
“The strapline below the headline proclaimed: "By far the most potent source of energy is gravity. Using it as power, future aircraft will attain the speed of light." It was written by one Michael Gladych…”
Gladych, portrayed by Cook as merely the random author of science journalism article, rapidly fades from the story, amidst numerous claims of government and aerospace industry conspiracies to cover-up the ‘true’ anti-gravity programs of the 1950s.  Ironically, Gladych is arguably a far more interesting, and ‘true-to-life’ character than any of the others in “The Hunt for Zero Point”.  Gladych, sometimes science journalist, sometimes biographer, and sometimes government operative enters our story as a fighter pilot ace of World War II; one of the few to be decorated by four different air forces; one who talked his way from America back to Europe to re-join the Allied forces, piloting fighter planes without any official military orders.  Gladych, whose life would turn out to be anything but quiet or private, makes his first public appearance in an article in the Syracuse Herald Journal of April 7th, 1944 titled, “Polish Fighter Leaves No Further Address”.  The story details how Gladych who traveled to the United States attached to an advanced Army flying school in New York, subsequently bluffed his way onto a clipper back to England, and joined an American squadron where he knew a few other Polish pilots.  He flew several missions before anyone thought to ask for the orders that he simply didn’t have.
A few years later in 1946, Gladych made nationally syndicated press rounds again, still a subject of the news rather than its author.  The headline read: “Ace to be Deported”[3].  After surviving several wartime adventures, not the least of which was the single-handed rescue of his younger brother Jan from a Soviet POW camp, Gladych had re-entered the United Sates, married, had a child, and settled down.  However, he’d entered the country under a ruse again, this time using a visa for Great Britain to gain access and was on the verge of being deported. The article went on to state that a grass-roots effort had formed to keep the Gladych family state-side: “Polish groups in the U. S. have joined veterans' organizations in an attempt to have the deportation order rescinded.”  Everything worked out for the best; the suspension of Gladych’s deportation was documented in the Senate Journal of March 7, 1950 [4]. 
Soon after avoiding deportation, Gladych transitionsed from gallivanting fighter pilot to prolific author.  The remaining sections of this article detail the scientific world of the ‘50s and ‘60s as seen through Gladych’s eyes.  Perhaps, not surprisingly, given Gladych’s lifestyle, his selection of scientific subject matter tended towards the legendary.  In each case, we’ll first look at the events through the lens of Gladych’s coverage.  Then we’ll take a wider view, or Gladych’s subjects, surveying their connections with and future impacts on both fringe and mainstream physics.

The GRF meetings
Other journalists that covered the meetings with similar accounts were xxx of the yyy Herald Tribune and xxx in American Modeler, a publication that Gladych also wrote for.
Mention Trimble and detail the Satellite magazine where the author points out that satellites and anti-gravity were both considered fringe science subjects.  Also point out the Witten connection.  
Bryce DeWitt makes anti-gravity cool.  Arnowitt and Deser and Openhemier story.
Project Smoke Puff
Even Gladych’s coverage of mainstream government funded science of the day had fringe-laden overtones.  Project Smoke Puff was obstensibly an effort to provide on-command over-the-horizon military communications.  The chief Stanford-based radio engineer worked with Dr. xxx Bracewell of Bracewell probe fame.  
In the same issue of the magazine was an article, complete with pictures of mutant plants, detailing scientific efforts to create giant peanuts through irradiation of their seeds.

Mainstream Afterglow
Mention the DeWitts, Dr. Witten, the junior Dr. Witten, the Apollo program, sky writing, 
No anti-gravity, but the origin of mass, the Higgs connection
Arnowitt and Deser 3+1 and super-gravity


Gladych, an Afterword
Gladych rescues his brother Jan from a Soviet POW camp
Stnaislaus architected the very FBI building where Mulder and Scully famously occupy the basement.
The three brothers live for a time in England
Gladych mentioned as an operative in Project Artichoke

Notes:  
