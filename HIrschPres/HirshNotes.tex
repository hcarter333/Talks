
\documentclass[prb,preprint]
{revtex4-1} 
% The line above defines the type of LaTeX document.
% Note that AJP uses the same style as Phys. Rev. B (prb).

% The % character begins a comment, which continues to the end of the line.

\usepackage{amsmath}  % needed for \tfrac, \bmatrix, etc.
\usepackage{amsfonts} % needed for bold Greek, Fraktur, and blackboard bold
\usepackage{graphicx} % needed for figures

\begin{document}

% Be sure to use the \title, \author, \affiliation, and \abstract macros
% to format your title page.  Don't use lower-level macros to  manually
% adjust the fonts and centering.

\title{Hirsch Presentation Notes}
% In a long title you can use \\ to force a line break at a certain location.

\author{Hamilton B. Carter}
%\email{hcarter333@tamu.edu} % optional
% optional second address
% If there were a second author at the same address, we would put another 
% \author{} statement here.  Don't combine multiple authors in a single
% \author statement.
\affiliation{Department of Physics, Texas A\&M University, College Station, TX 77843}
% Please provide a full mailing address here.


% See the REVTeX documentation for more examples of author and affiliation lists.

\date{\today}

%\begin{abstract}


%\end{abstract}
% AJP requires an abstract for all regular article submissions.
% Abstracts are optional for submissions to the "Notes and Discussions" section.




%\maketitle % title page is now complete

%\newpage
%\section{Board 1}

%\begin{figure}[h!]
%\centering
%\includegraphics[width=5in]{board1_2014_06_24.jpg}
% Notice the width specification.  Photographs should normally have a
% resolution of approximately 300 pixels per inch when printed, that is,
% a total width of about 1000 pixels for a photo to be printed one column
% wide.  Note also that this included photo is in .jpg format even though 
% a .tiff version should be submitted for final production.
%\caption{Board 1)}
%\label{Board 1}
%\end{figure}
\centerline{\bf Hirsch Presentation Notes}
\bigskip

Summary:  
\\
Notes on the spin velocity an dother dynamics are taken from 
\\
\url{https://drive.google.com/file/d/0B30APQ2sxrAYRWFJTy1pZlp4Qzg/edit?usp=sharing}
\\
Hirsch, J. E., "KINETIC ENERGY DRIVEN SUPERCONDUCTIVITY AND SUPERFLUIDITY", Modern Physics Letters B, Vol. 25, No. 29 (2011) 2219–2237
\\
\\
Here is the text to be associated with the literagture review LitReview Spin Velocity.  First off, $k_F$ is the Fermi wave vector.
\\
"The theory predicts that as the system goes superconducting an expansion
of the electronic wavefunction occurs, driven by kinetic energy lowering, from
linear extension $k^{−1}_F$
($k_F$ = Fermi wavevector) in the normal state, to $2\lambda_L$ in
the superconducting state, with $2\lambda_L$ the London penetration depth.[references]5,20."
\\
Here's how it all happens
\\
As the sample goes superconducting, the paired hole orbits expand.  The Larmore susceptibliity is given by $\xi_{Larmor}\left( r = 2\lambda_L\right) = -\dfrac{n e^2}{4 m_e c^2}\left(2 \lambda_L\right)^2 = -\dfrac{1}{4\pi}$
\\
This indicates perfrect diamagnetism.
\\
We now have paired electrons in orbits that are counterrotating with opposite spins, 
\\
"In the $2 \lambda_L$ orbits, opposite spin electrons traverse their orbits in opposite
directions."
\\
and here's the money statement!
\\
"The electrons acquire their orbital speed through the spin-orbit
interaction of the magnetic moment of the electron with the ionic charge background
as the orbits expand from the microscopic scale ($k^
{−1}_F$) ($k_F$ = Fermi wavevector) to
$2 \lambda_L$, through a “quantum spin Hall effect”. [reference] 5"
\\
Here's where it's derived!!!
\\
"The speed of the electrons is derived
from the strength of the spin orbit interaction as given by Dirac’s equation with
the electric field generated by the ionic background and yields5,22
"
\\
Here's the expression we start from
\\
$v_\sigma^0 = -\dfrac{\hbar}{4 m_e \lambda_L}\hat{\sigma} \times \hat{n}$
\\
The orbital angular momentum o the electrons in the $2 \lambda_L$ orbits is 
\\
$L = m_e v_\sigma^0 \left(2 \lambda_L\right) = \dfrac{\hbar}{2}$
\\
The spin current becomes macroscopic
\\
"The superposition of these orbital motions gives rise to a macroscopic spin current within a London penetration depth of the surface of superconductors, parallel to the surface, [references] 23.  This is a macroscopic zero point motion of the superconductor with is predicted to exist even at zero temperature [ref] 5."
\\
More great stuff that needs to be quoted leading to equation 6, To Do, go back and get this stuff.  Then, we get the next money shot,
\\
$B_s = \dfrac{\hbar c}{4 |e| \lambda^2}$
\\
This is the expression for the critical field in terms of the London penetration depth.  
\\
Oooohhh, this isn't the greatest thing in the world.  The spin current velocity is not dependnt on the size of the sample, so how can the x-ray energy be dependent on the size of the sample?
\\
\\
Spin current velocity, equation (5)
$v_\sigma = v_\sigma^0 \pi \dfrac{e}{m_e c}B \lambda_L$
\\Bummer, no velocity related to size of the superconductor in this article!!!
\\
\\
Here's where the velocity of the particle comes in relation to the x-ray energy.  To Do:  Be sure to ask Hirsch how the spin velocity from the other article, in Joural of Physics: condensed Matter above, reltes to the orbital velocity around the sample here.  Does it relate at all?  Also asks how this jibes with the new energy equation he sent and where and how he derived that.  The article being discussed now is at 
\\
\url{https://drive.google.com/file/d/0B30APQ2sxrAYaTIyNjJaZ0Q0TUE/edit?usp=sharing}
\\
In the "Ionizing radiation from superconductors" article we do have that 
\\
"As the size of the sample becomes larger the amount of expelled charge increases, and we argue that when the potential energy of an electron at the edge of the positive charge distribution becomes larger in magnitude than twice the electron rest energy... "
\\
Then we get equation six which is based on charge and radius.
\\
Better though, is equation 7 which finally has the velocity defined as the size of the sample.  Mind you here that it's not necessarily the spin current velocity here though, just the velocity of the electron orbiting the entire sample.  Here's the formula, and quote, 
\\
"Alternatively, we may argue that the condition of dynamic equilibrium for an electron
orbiting at radius r in the field of a positive charge (−q)"
\\
$\dfrac{m_e v^2}{r} = \dfrac{qe}{r^2}$
\\
Which, according to Hirsch shows that $v$ approaches the speed of light for $\dfrac{q e }{r} = m_e c^2$
\\
OK, now we have a velocity based on 
\\\\\\
Critical field in terms of spin current velocity
$B_s = \dfrac{\hbar c}{4 | e | \lambda_L^2}$


$\vec{\nabla} \cdot \left(\vec{\nabla} \times \vec{A} \right) = 0$
\\
\\





\newpage
\section{Transition to Perfect Diamagnetism}
\bigskip

Hirsch says:
\\
"To obtain the perfect diamagnetism of superconductors, the diamagnetic susceptibility of n electrons per unit volume has to take the value $\dfrac{-1}{\left(4\pi\right)}$, hence
\\
$\dfrac{-1}{\left(4\pi\right)} = -\dfrac{n e^2}{4 m_e c^2} \left< r_\perp^2 \right>$.
\\
Which will be the case when \\
$\sqrt{\left< r_\perp^2 \right>} = 2\lambda_L$
\\
Note:  $\lambda_L$ is defined in Hirsch's reference 47 which is Tinkham.

\newpage
\section{Next Steps 2014/08/23}
\bigskip

Look into the Larmor suscetpibility material
\\
\\
\newpage
\section{Notes on the hopping interaction}
\bigskip
Equation 3 is the hopping amplitude
\\
$t_{ij} = t_h + \Delta t\left(n_{i,-\sigma} + n_{j,-\sigma}\right)$
\\\\
Equation 5 is the hopping interaction, (mine: in the Hamiltonian)
\\
$V_h = -\Delta t \sum {\left< ij \right> }$

\newpage
\section{Assorted Notes}
\bigskip
\url{https://drive.google.com/file/d/0B30APQ2sxrAYRWFJTy1pZlp4Qzg/edit?usp=sharing}
\\
The formula for gain from attenuation can be derived using
\\
$dB = 20 log\left(gain\right)$
\\
$gain = exp\left\{2.303 \dfrac{dB}{20} \right\}$, 
\\
Where 2.303 is the conversion factor using log base 10 instead of the natural logarithm.
\\
Notes for channel gain
\\
\\
$\dfrac{320}{1024} = \dfrac{662}{channel}$
\\
\\
Solving, we get that the channel for 662 peak is $2118$.
\\
The gain can be determined using
\\
\\
$peak_{662} = 2771.3 gain - 89.457$
\\
\\
Solving this gives a gain of $0.79$, which translates to about 3 dB.
\\
\\
$\dfrac{ping\ frequency}{listening\ frequency} =\dfrac{meteor\ velocity}{speed\ of\ light}$
\\
\newpage
\section{YBCO attenuation}
\bigskip
Data are from
\\
\url{http://www.nist.gov/pml/data/xraycoef/}
\\\\
The attenuation formula is
\\
$\dfrac{I}{I_0} = exp\left[ -\left(\dfrac{\mu}{\rho}\right)x\right]$
\\
The tables contain data in terms of $\dfrac{\mu}{\rho}$ for various materials.

Air is hard on soft x-rays at least around 1.0 keV or so, and even up to 3 keV.  Even at ranges like 1 cm, the intensity scaling defined above is $1.7 \times 10^{-58}$.  Going to a half centimeter doesn't help in any appreciable way.  The atteuation factor is $1.3 \times 10^{-29}$.
\\\\
Whoops, it's not nearly as bad as it originally appeared.  The value x shown above is not the distance, but the mass thickness!  That is, $x = distance \times density$ where density is in $g/cm^3$ and the distance is in $cm$.  Using this value, the lower end of the scale at 1 keV is kind of puny, but the rest of the spectrum up to 3 keV is all right and essentially transparent with an $I/I_0$ of $0.90$
\\\\
Styrofoam is not the insulator we're looking for.  It has a rather high density and a rather high $\mu / \rho$.  At half a cm, $I/I_0 = 7.8 \times 10^-19$.
\\\\
Pyrex is also a poor choice with a single mm attenuating 3 keV x-rays by $I/I_0 = 2.4 \times 10^-50$.
\\\\
Slipping back into lit review mode, the following is taken from\\ \url{http://en.wikipedia.org/wiki/Photographic_hypersensitization}
\\
and echoes what I was told at lunch today.  The particular two subjects of interest are pre-flashing to make the photo more sensitive, as well as exposure at low temeperature to find out what the emulsion does at low temperature.  First, the quote from the article
\\\\
"Practical, user-applied hypersensitizing techniques have evolved over most of the last century and fall mostly into four types of treatments. Broadly, these involve liquid phase (washing), gas phase (out-gassing and baking and hydrogenation), exposure at lowered temperature, and pre-flashing. Some of these can be used in combination, but many severely shorten the shelf-life of a product and so can not be applied by the manufacturer."
\\\\
The temperature sensitivity led to a new reference \\
\url{http://www.opticsinfobase.org/josa/abstract.cfm?uri=josa-25-1-4}
\\
which contins information on x-ray sensitivity at near liquid nitrogen temperatures.
\\\\
This reference contains a calibration of Kodak bio film and hopefully goes down to the number of photons that can be detected.  Looks very handy.
\\
\url{http://scitation.aip.org/content/aip/journal/rsi/77/10/10.1063/1.2221698}
\\
\\
\\
\newpage
\section{Spin and Sample Orbits}
\bigskip
On the one hand Hirsch is talking about $v_\sigma$ on the other $v_{orbit}$ for the whole sample.  $v_{orbit}$ exists so the charges donot fallback into the field center.  $v_\sigma$ exists as spin-orbit coupling.  $v_\sigma$ is defined by $\xi$ and or is defined by ? $H_c$.  $v_{orbit}$ is defined by the net excess charge density and the associated electric field  They are related, the hole - copuling defines the spin orbit parameters.
\\
Fewer holes lead to even wider  wave functions????  Check this.
\\
It is the width of thse wave functions $k_F^{-1}$ that defines or at least is a parameter of $\rho_-$.  $\rho_-$ defines the charge density and isn turn demands the required sample orbital $v_{orbit}$ which provides the energy for the x-rays.  The entire macroscopic spin wave function orbits in the field made by $\rho_-$.

%It looks like board 2 is just a better quality photo of board 1.

%\begin{figure}[h!]
%\centering
%\includegraphics[width=5in]{board2_2014_06_24.jpg}
% Notice the width specification.  Photographs should normally have a
% resolution of approximately 300 pixels per inch when printed, that is,
% a total width of about 1000 pixels for a photo to be printed one column
% wide.  Note also that this included photo is in .jpg format even though 
% a .tiff version should be submitted for final production.
%\caption{Board 2)}
%\label{Board 2}
%\end{figure}

%\newpage
%\section{board 3}

%\begin{figure}[h!]
%\centering
%\includegraphics[width=5in]{board3_2014_06_24.jpg}
% Notice the width specification.  Photographs should normally have a
% resolution of approximately 300 pixels per inch when printed, that is,
% a total width of about 1000 pixels for a photo to be printed one column
% wide.  Note also that this included photo is in .jpg format even though 
% a .tiff version should be submitted for final production.
%\caption{Board 3)}
%\label{Board 3}
%\end{figure}


%Board 13

%\begin{figure}[h!]
%\centering
%\includegraphics[width=5in]{board13_06_19_2014.JPG}
% Notice the width specification.  Photographs should normally have a
% resolution of approximately 300 pixels per inch when printed, that is,
% a total width of about 1000 pixels for a photo to be printed one column
% wide.  Note also that this included photo is in .jpg format even though 
% a .tiff version should be submitted for final production.
%\caption{Board 13}
%\label{Board 13}
%\end{figure}

%Board 14

%\begin{figure}[h!]
%\centering
%\includegraphics[width=5in]{board14_06_19_2014.JPG}
% Notice the width specification.  Photographs should normally have a
% resolution of approximately 300 pixels per inch when printed, that is,
% a total width of about 1000 pixels for a photo to be printed one column
% wide.  Note also that this included photo is in .jpg format even though 
% a .tiff version should be submitted for final production.
%\caption{Board 14}
%\label{Board 14}
%\end{figure}











%Board 4a

%\begin{figure}[h!]
%\centering
%\includegraphics[width=5in]{board4a_2014_06_12.jpg}
% Notice the width specification.  Photographs should normally have a
% resolution of approximately 300 pixels per inch when printed, that is,
% a total width of about 1000 pixels for a photo to be printed one column
% wide.  Note also that this included photo is in .jpg format even though 
% a .tiff version should be submitted for final production.
%\caption{Board 4a}
%\label{Board 4a}
%\end{figure}





% If your manuscript is conditionally accepted, the editors will ask you to
% submit your editable LaTeX source file.  Before doing so, you should move
% all tables and figure captions to the end, as shown below.  Tables come 
% first, followed by figure captions (with figure inclusions commented-out).
% Figures should be submitted as separate files, collected with the
% LaTeX file into a single .zip archive.

%\newpage   % Start a new page for tables

%\begin{table}[h!]
%\centering
%\caption{Elementary bosons}
%\begin{ruledtabular}
%\begin{tabular}{l c c c c p{5cm}}
%Name & Symbol & Mass (GeV/$c^2$) & Spin & Discovered & Interacts with \\
%\hline
%Photon & $\gamma$ & \ \ 0 & 1 & 1905 & Electrically charged particles \\
%Gluons & $g$ & \ \ 0 & 1 & 1978 & Strongly interacting particles (quarks and gluons) \\
%Weak charged bosons & $W^\pm$ & \ 82 & 1 & 1983 & Quarks, leptons, $W^\pm$, $Z^0$, $\gamma$ \\
%Weak neutral boson & $Z^0$ & \ 91 & 1 & 1983 & Quarks, leptons, $W^\pm$, $Z^0$ \\
%Higgs boson & $H$ & 126 & 0 & 2012 & Massive particles (according to theory) \\
%\end{tabular}
%\end{ruledtabular}
%\label{bosons}
%\end{table}

%\newpage   % Start a new page for figure captions

%\section*{Figure captions}

%\begin{figure}[h!]
%\centering
%\includegraphics{GasBulbData.eps}   % This line stays commented-out
%\caption{Pressure as a function of temperature for a fixed volume of air.  
%The three data sets are for three different amounts of air in the container. 
%For an ideal gas, the pressure would go to zero at $-273^\circ$C.  (Notice
%that this is a vector graphic, so it can be viewed at any scale without
%seeing pixels.)}

%\label{gasbulbdata}
%\end{figure}

%\begin{figure}[h!]
%\centering
%\includegraphics[width=5in]{ThreeSunsets.jpg}   % This line stays commented-out
%\caption{Three overlaid sequences of photos of the setting sun, taken
%near the December solstice (left), September equinox (center), and
%June solstice (right), all from the same location at 41$^\circ$ north
%latitude. The time interval between images in each sequence is approximately
%four minutes.}
%\label{sunsets}
%\end{figure}

\end{document}
